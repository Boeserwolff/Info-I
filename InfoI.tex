\documentclass[a4paper,12pt]{article}
\usepackage[top=2cm, bottom=2cm, left=3cm, right=3cm, headsep=0mm, footskip=10mm]{geometry}
\usepackage[ngerman]{babel}
\usepackage[T1]{fontenc}  % Silbentrennung bei Sonderzeichen
\usepackage[utf8]{inputenc} %Umlaute & Co
\usepackage[normalem]{ulem}
\usepackage{mathtools,amsthm,amssymb,amsfonts}
\usepackage{marvosym}
\usepackage{gauss} % Gaussverfahren mit Pfeilen
\usepackage[table,xcdraw]{xcolor}
\usepackage{graphicx}
\usepackage{tikz}
\usepackage{csquotes}
\usepackage{fancyhdr}
\usepackage{amssymb}
\usepackage[neverdecrease,defblank]{paralist}   % enumerate, itemize,...
\usepackage[shortlabels]{enumitem}
\usepackage{titlesec}
\usepackage{bm} %Fette Mathezeichen
\usepackage[onehalfspacing]{setspace}
\usepackage{array}
\usepackage{ifthen}
\usepackage{xparse}
\usepackage{etoolbox} % toolbox of programming tools
\usetikzlibrary{trees,automata,arrows,shapes}
\usetikzlibrary{shapes.geometric}
\newcommand{\warningsign}{\tikz[baseline=-.75ex] \node[shape=regular polygon, regular polygon sides=3, inner sep=0pt, draw, thick] {\textbf{!}};}

\begin{document}
\begin{titlepage}
\centering
	{\scshape\Large Eberhard Karls Universität Tübingen \par}
	\vspace{1cm}
	{\scshape\Large \par}
	\vspace{1.5cm}
	{\huge\bfseries Informatik I Vorlesung\par}
	\vspace{2cm}
	{\Large Wintersemester 2016/2017 \par}
	\vfill
	Mitschrieb von \par Julian Wolff

	\vfill

% Bottom of the page
	{\large Aktueller Stand \today\par}
\end{titlepage}
\tableofcontents
\newpage
\section{Scheme: Ausdrücke, Auswertung und Abstraktion}
\subsection{REPL}
\begin{center}
\begin{tabular}{|l|c|}
\hline
Definition&DrRacket\\
\hline
Interaktion&REPL\\
\hline
\end{tabular}
\end{center}

Die Anwendung von Funktionen wird in Scheme \underline{ausschließlich} in \underline{Präfixnotation} durchgeführt:
\begin{center}
\begin{tabular}{|cc|}
\hline
Mathematik & Scheme\\
\hline
44-2 & (-44 2)\\
f(x,y) & (f x y)\\
$\sqrt{81}$ & (sqrt 81)\\
& (floor x)\\
$9^2 $& (expt 9 2)\\
3! & (! 3)\\
\hline
\end{tabular}\\

Allgemein: $\langle $Funktion$\rangle \langle$ argument$\rangle$)
\end{center}
(+ 402 ) und (odd? 42) sind Beispiele für die \underline{Ausdrücke}, die bei Auswertung einen Wert liefern. (Notation $\rightsquigarrow$) heißt Auswertung/Evaluation/Reduktion.\\
(+ 40 2) $\underset{Eval}{\rightsquigarrow}$ 42\\
(add? 42)$\underset{Eval}{\rightsquigarrow}$ \#f\\
\\
Interaktionsfenster:\\ 
Re$\underset{\text{\large{Loop}}}{\underbracket{\text{ad}   \rightsquigarrow \text{Eval}  \rightsquigarrow \text{Pri} }}$nt      \hfil     REPL
\\
\\
\subsection{Literale}
\uline{Literale} stehen für einen konstanten Wert (auch: \underline{Konstante}) und sind nicht weiter reduzierbar.
\begin{center}\begin{tabular}{ccl}
\underline{Literal} & &\underline{Signatur}\\
\#t \#f & (true, false, Wahrheitswerte) & boolean\\
\enquote{ac} \enquote{x} \enquote{ } & (Zeichenketten) & string\\
0 1904 -42 007 & (ganze Zahlen) & integer\\
0.42 3.1415 -273.15 & (Fließkommazehlen) & real\\
1/2 3/4 -1/10 & (rationale Zahlen) & rational\\
 \includegraphics[width=0.17\textwidth]{frod.jpg}& (Bilder) & image\\
\end{tabular}
\end{center}
Auswertung \underline{zusammengesetzte Ausdrücke} (composite expression) in mehreren Schritten (Steps), "von innen nach außen" , bis keine weitere Reduktion möglich ist:\\
(+(+20 20)(+ 1 1)) $ \rightsquigarrow$ (+ 40 (+ 1 1))  $ \rightsquigarrow$ (+ 40 2)  $ \rightsquigarrow$ 42\\
\\
\underline{Beispiel:}\\
$0.7+(\frac{1}{2}/0.25)-(0.6/0.3)=0.7$\\
\\
\warningsign\underline{Achtung:} Scheme rundet bei Arithmetik mit Fließkommazahlen (interne Darstellung nicht präzise). Die Arithmetik mit rationalen Zahlen ist exakt.\\
\subsection{Identifier}
Ein Wert kann an einen \underline{Namen} (identifier) \underline{gebunden} werden, durch(define$ \langle$id$ \rangle \langle$expression$\rangle$)\\
Es erlaubt konsistente Wiederverwendung und dient der Selbstdokumentation von Programmen.\\
\\
\warningsign \uline{Achtung:} Dies ist eine \underline{Spezialform} und kein Ausdruck. Insbesondere besitzt diese Spezialform keinen Wert, sondern einen Effekt: der Name $\langle$id$\rangle$ wird durch den Wert von $\langle $expression$\rangle$ \underline{gebunden}. Namen können in Scheme fast beliebig gewählt werden, solange
\begin{itemize}
\item[$\bullet$]die Zeichen $ () \lbrack \rbrack   \{\} ", '; \# \backslash$ | nicht vorkommen
\item[$\bullet$]der name nicht einem numerischen Literal gleicht
\item[$\bullet$] keinen Whitespaße (Leerzeichen, Tabulatoren, Neuwlines) enthalten sind
\end{itemize}
\uline{Beispiel:} Euro $\rightarrow$ US-\$\\ \\
\warningsign \uline{ Achtung:} Groß-/Kleinschreibung ist in Identifiern \underline{nicht} relevant.\\



\end{document}
